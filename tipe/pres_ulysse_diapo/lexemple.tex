\documentclass[10pt]{beamer}

\usepackage[utf8]{inputenc}
\usepackage[T1]{fontenc}
\usepackage{lmodern}
\usepackage{amsmath, amssymb}
\usepackage{graphicx}
\usepackage{stmaryrd}
\usepackage{babel}


%CHOIX DU THEME et/ou DE SA COULEUR
% => essayer différents thèmes (en décommantant une des trois lignes suivantes)
\usetheme{PaloAlto}
%\usetheme{Madrid}
%\usetheme{Copenhagen}

% => il est possible, pour un thème donné, de modifier seulement la couleur
%\usecolortheme{crane}
%\usecolortheme{seahorse}

%\useoutertheme[left]{sidebar}
%Pour le TITLEPAGE

\title{Cryptographie: De la théorie à la pratique}
\subtitle{Application au vote en ligne}
\author[Beaujeu Amélien]{Beaujeu Amélien}
\date{2020-2021}
\institute[UT3 -- FSI]{MPX-Lycée Blaise Pascal}


\begin{document}

\frame{\titlepage}

\begin{frame}
\frametitle{Sommaire}
\begin{enumerate}
\item \underline{Présentation}
\begin{enumerate}
\item Enjeux et sécurité
\item Présentation architecture
\end{enumerate}
\item \underline{Protocoles}
\begin{enumerate}
\item RSA
\item El Gamal
\end{enumerate}
\item \underline{Bilan et mise en perspective}
\begin{enumerate}
\item Performances et adaptation
\item Vote en ligne: une possibilité à envisager?
\end{enumerate}
\end{enumerate}
\end{frame}

\begin{frame}
\frametitle{Présentation}
\begin{center}
\includegraphics[scale=0.4]{CNIL.jpg}
\end{center}
https://www.cnil.fr/fr/securite-des-systemes-de-vote-par-internet-la-cnil-actualise-sa-recommandation-de-2010
\end{frame}

\begin{frame}
\frametitle{Présentation}
\begin{center}
\underline{Notre architecture}
\includegraphics[scale=0.45]{Architecture.jpg}
\end{center}
\end{frame}

\begin{frame}
\frametitle{Protocoles}
\framesubtitle{Protocoles étudiés et base de fonctionnement}
\begin{enumerate}
\item RSA
\item El Gamal
\end{enumerate}
Ces deux protocoles sont basés sur un chiffrement \underline{asymétrique}.
\end{frame}

\begin{frame}
\frametitle{Protocoles}
\framesubtitle{RSA-Histoire et utilisations actuelles}
\begin{enumerate}
\item \underline{Histoire}
\begin{itemize}
\item 1977 par Ronald Rivest, Adi Shamir et Leonard Adleman
\end{itemize}
\item \underline{Utilisations actuelles}
\begin{itemize}
\item Authentification dans l'échange d'informations chiffrées
\item Le chiffrement des clés symétriques utilisées lors du reste du processus d'échange d'informations numériques chiffrées
\end{itemize}
\end{enumerate}
\end{frame}

\begin{frame}
\frametitle{Protocoles}
\framesubtitle{RSA-Protocoles}
\begin{enumerate}
\item \underline{Création des clés}

$ p \hspace{0.1cm} et \hspace{0.1cm} q \in \mathcal{P} \hspace{0.4cm} n=p\times q$\\
$ e \in \mathbb{N}^{*}\hspace{0.1cm} / \hspace{0.1cm} e\wedge  \varphi(n)=1$\\
$Calculer \hspace{0.1cm} d \in \mathbb{Z}\hspace{0.1cm}/\hspace{0.1cm}\hspace{0.1cm}de=1\hspace{0.1cm}\left[\varphi(n)\right]$\\
$(e,n)$ est la clé publique, $d$ est la clé privée de l'utilisateur.
\item \underline{Chiffrement}\\
Message \hspace{0.1cm} $M<n \hspace{0.1cm} et \hspace{0.1cm} M^{e}=C\hspace{0.1cm}[n]$\\
C est alors le message chiffré
\item \underline{Déchiffrement}\\
$M=C^{d}\hspace{0.1cm}[n]$
\end{enumerate}
\end{frame}

\begin{frame}
\frametitle{Protocoles}
\framesubtitle{RSA-Problèmes rencontrés et complexité}
\begin{enumerate}
\item Génération des nombres premiers
\begin{enumerate}
\item Méthode naïve
\item Algorithme probabiliste de Miller-Rabin
\end{enumerate}
\item Génération d'un nombre premier avec \hspace{0.1cm}$\varphi(n)$
\begin{itemize}
\item Faux problème car probabilité que deux nombres soient premiers entre eux (Théorème de Césaro) vaut\hspace{0.1cm}$\frac{6}{\pi^{2}}$\\
\end{itemize}
\item Exponentiation modulaire
\begin{enumerate}
\item Méthode naïve
\item Exponentiation rapide
\end{enumerate}
\item Obtention de courbes de complexité temporelle expérimentale
\end{enumerate}
\smallskip
\begin{figure}[h]
\centering
\begin{minipage}[c]{.46\linewidth}
\centering
\includegraphics[scale=0.3]{PLOTRSA1.png}
\end{minipage}
\hfill%
\begin{minipage}[c]{.46\linewidth}
\centering
\includegraphics[scale=0.3]{PLOTRSA2.png}
\end{minipage}
\end{figure}
\end{frame}

\begin{frame}
\frametitle{Protocoles}
\framesubtitle{RSA-Résultat de notre code}
\begin{center}
\includegraphics[scale=0.6]{RSACODE.jpg}
\end{center}
\begin{center}
\includegraphics[scale=0.8]{RSARES.jpg}
\end{center}
\end{frame}

\begin{frame}
\frametitle{Protocoles}
\framesubtitle{ElGamal-Histoire et utilisations actuelles}
\begin{enumerate}
\item \underline{Histoire}
\begin{itemize}
\item 1984 Par Taher ElGamal
\end{itemize}
\item \underline{Utilisations actuelles}
\begin{itemize}
\item Cryptosystèmes basés sur ElGamal sur le groupe des courbes elliptiques
\end{itemize}
\end{enumerate}
\end{frame}

\begin{frame}
\frametitle{Parenthèse}
\framesubtitle{DLP-Problème du logarithme discret}
\begin{block}{Définition}
Soit $\mathcal{G}$\hspace{0.05cm}un groupe cyclique d'ordre \hspace{0.05cm}$n$\hspace{0.05cm} engendré par \hspace{0.05cm}$g$\hspace{0.05cm}. Pour\hspace{0.05cm}$y\in \mathcal{G}$\hspace{0.05cm}, trouver\hspace{0.05cm}$k\in \mathbb{Z}\hspace{0.05cm}/g^{k}=y$\hspace{0.05cm} est le problème du logarithme discret en base g.\\
On note\hspace{0.05cm}$k=log_{g}y$\hspace{0.05cm}, bien défini modulo\hspace{0.05cm}$n$. 
\end{block}
\begin{block}{Théorème [Shoup, 1997]}
Dans un groupe générique d'ordre \hspace{0.05cm}$n$\hspace{0.05cm} premier, la résolution du DLP nécessite au moins\hspace{0.05cm}$\sqrt{n}$\hspace{0.05cm} multiplications.
\end{block}
\begin{alertblock}{Conjecture}
Résoudre le problème de Diffie-Hellman, $ie$ trouver \hspace{0.05cm}$g^{k_{A}k_{B}}$\hspace{0.05cm} en connaissant \hspace{0.05cm}$g,g^{k_A},g^{k_B}$\hspace{0.05cm} est aussi difficile que de résoudre le DLP dans \hspace{0.05cm}$\mathcal{G}$.
\end{alertblock}
\end{frame}

\begin{frame}
\frametitle{Protocoles}
\framesubtitle{ElGamal-Principe de fonctionnement par Diffie-Hellman}
\begin{itemize}
\item \underline{Génération clé publique et clé privée}\\
On prend un groupe cyclique\hspace{0.05cm}$\mathcal{G}\left(=\mathbb{F}_q\right)$\hspace{0.05cm} en pratique, engendré par \hspace{0.05cm}$g$, d'ordre \hspace{0.05cm}$q$.\\
On choisit \hspace{0.05cm}$x \in \llbracket1;q-1 \rrbracket$. C'est la clé privée de l'utilisateur.\\
$h=g^{x}$\\
$\left(\mathcal{G},q,g,h\right)$\hspace{0.05cm} est la clé publique.
\end{itemize}
\end{frame}

\begin{frame}
\frametitle{Protocoles}
\framesubtitle{ElGamal-Principe de fonctionnement par Diffie-Hellman}
\begin{itemize}
\item \underline{Cryptage}\\
Alice veut envoyer un message\hspace{0.05cm}$\mathcal{M}$\hspace{0.05cm}sécurisé à Bob.\\
Elle associe $\mathcal{M}\rightarrow m,\hspace{0.05cm}m \in \mathcal{G}$\\
Alice choisit \hspace{0.05cm}$y \in \llbracket1;q-1 \rrbracket$, puis calcule:\\
$s=h^{y}=g^{xy}$, appelé secret commun.\\
Puis \hspace{0.05cm}$c_1=g^{y}$\hspace{0.05cm}et\hspace{0.05cm}$c_2=ms=mg^{xy}$.\\
Alice envoie\hspace{0.05cm}$(c_1,c_2)$\hspace{0.05cm}à Bob.
\end{itemize}
\end{frame}

\begin{frame}
\frametitle{Protocoles}
\framesubtitle{ElGamal-Principe de fonctionnement par Diffie-Hellman}
\begin{itemize}
\item\underline{Décryptage}\\
Bob reçoit\hspace{0.05cm}$(c_1,c_2)$\hspace{0.05cm} et possède sa clé privée.\\
Il calcule alors\hspace{0.05cm}$c_1^{x}=g^{xy}=s$.\\
Alors en possession de\hspace{0.05cm}$s$\hspace{0.05cm}, il calcule\hspace{0.05cm}$s^{-1}=c_1^{q-x}$.\\
Et donc,\hspace{0.05cm}$c_2c_1^{q-x}=mss^{-1}=me=\boxed{m}$
\end{itemize}
\end{frame}

\begin{frame}
\frametitle{Protocoles}
\framesubtitle{ElGamal-Problèmes rencontrés}
\begin{enumerate}
\item Exponentiation modulaire
\begin{itemize}
\item Exponentiation rapide-idem RSA
\end{itemize}
\item Génération du résultat comme nombre
\begin{figure}
\begin{minipage}[t]{6cm}
\centering
\includegraphics[scale=0.4]{ELGLIST.jpg}
\end{minipage}
\begin{minipage}[t]{6cm}
\centering
\includegraphics[scale=0.34]{ELGRES.jpg}
\end{minipage}
\end{figure}
\item Problème du générateur: sécurité
\begin{center}
\includegraphics[scale=0.4]{ELGG.jpg}
\end{center}
\end{enumerate}
\end{frame}

\begin{frame}
\frametitle{Protocoles}
\framesubtitle{ElGamal-Complexité algorithmique}
\begin{center}
\includegraphics[scale=0.5]{PLOTELG.png}
\end{center}
\end{frame}

\begin{frame}
\frametitle{Bilan et mise en perspective}
\framesubtitle{Protocole adapté et faisabilité}
\begin{itemize}
\item ElGamal bien plus performant à notre niveau + possibilité de faire encore plus sécurisé ECC
\item Problème de gestion
\vspace{1cm}
\begin{block}{Conclusion}
Le système avec une telle architecture est faisable à petite échelle. Néanmoins, au niveau d'un pays, garantir anonymat, intégrité et fiabilité semble être très compliqué, surtout en vertu de la quantité de calcul et de ressources nécessaire.
\end{block}
\end{itemize}
\end{frame}
\end{document}